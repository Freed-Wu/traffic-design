% !TEX root = ../main.tex
\fancypagestyle{plain}{\pagestyle{fancy}}
\setheader{专利说明书}
\maketitle
\section{技术领域}
	本实用新型公开了一种适用于车站、景点的出租车或大巴车换乘、拼车系统。
\section{技术背景}
	现如今,随着人们生活水平的提高,乘坐高铁、动车等便利的交通工具已经越来越受到普及。在一些人流量非常大的车站,每当列车到站后,车站便会变得非常拥堵,采用一些手段快速疏散这些人流便成为了需要解决的问题。从节约资源和绿色环保的角度来看,充分利用好公共汽车、地铁、出租车等资源便成为了解决该问题所要首先考虑到的。从乘坐出租车的角度来看,往往出租车所携带的乘客往往只有一两个,使得资源浪费比较大,与此同时,乘客单独乘坐也使得他的开销很高昂。考虑到使得出租车资源得到充分的利用以及尽量使得乘客花最少的钱坐最舒适的车,可以设计一种适用于车站、景点的出租车或大巴车换乘、拼车系统。
\section{技术内容}
	\subsection{技术目的}
		用于火车站或高铁站的出租车换乘点或旅游景点的观光车或大巴车乘坐点。提高服务效率,减少等待时间。
	\subsection{技术方案}
		如图\ref{外观}所示,最下方是车辆入口,车辆从地下通道或与等候席垂直的方向到达;车辆入口的上方设置有显示屏,显示接下来的车辆到达时间和乘坐该车辆的乘客的排队号;显示屏前方是等候席,供排队乘客等待。两侧可放置若干台终端机器供手机无法使用的乘客操作,界面与手机端相似,二维码可贴放在机器附近醒目的地方。
	\subsection{技术效果}
		有效减少了乘客的排队等待时间。为拼车的乘客节省车费的同时,减少了后续乘客的等待时间。通过已有的乘车数据计算出预计的等待时间,有利于减少乘客等待的焦躁情绪。与车站、景点App结合的方式起到了一定的宣传作用。
\section{附图说明}
	\renewcommand\listfigurename{\tiny}
	\vspace{-2em}
	\listoffigures
\section{实施方式}
	下面结合图\ref{程序流程图}具体说明对本方案的实施方式作进一步详细的描述。
	\par\indent 整个拼车系统主要包括以下几个模块:
	\begin{enumerate}
		\item 注册和登录模块:当乘客有乘车需求时,通过下载已开发出的APP或者通过微信扫描二维码进行本系统的账号注册,注册后即可登录并享用整个拼车系统的所有功能。
		\item 信息采集模块:此模块主要用于当乘客登录系统后,对乘客的需求信息进行采集,并将这些需求信息建立在一个数据库中。此模块除了采集乘客信息,还采集车辆信息,即车辆的到达时间、容量以及是否已经被预约等信息。
		\item 拼车匹配模块:此模块用于实现,当系统采集完乘客的需求信息之后,自动筛选出在规定时间段内的车辆信息。以供乘客进行选择;
		\item 乘客选择和确认模块:乘客在此模块可以根据自己的意愿确认所需拼车的类型和车号。系统生成乘客的排队号以及将此信息显示在候车室中,候车室在车辆到达前的十分钟会通过手机和现场提醒。若乘客仍未确定拼车车辆,则回到“拼车匹配模块”继续匹配;
		\item 支付模块:当乘客完成对车辆信息的匹配和确认后,系统根据计算生成乘客应付款的金额,乘客可以选择在线支付也可选择线下支付;
		\item 反馈模块:乘客完成拼车以后,从搭乘车辆到抵达目的地之间,都可以在系统上进行信息的反馈,比如在紧急情况下进行报警、对司机状态的评价以及整个拼车过程的乘客体验反馈。
	\end{enumerate}	
	\par\indent 整个拼车系统会通过APP和车站候车室放置二维码供乘客使用,同时也会在提供线上服务台供没带手机的乘客使用。
	\par\indent 乘客通过注册和登录后,首先对自己的需求及信息进行填写,此内容主要包括目的地、乘客人数、行李个数及类型、是否选择拼车等必要的信息。系统将用户的信息添加到系统中的信息表中,以进行和即将到站的车辆匹配。
	\par\indent 如果乘客选择拼车,系统根据已有的车辆到站信息以及已经被其他乘客预约的情况,自动匹配出和当前乘客目的地距离较近或在同一条线路上(车辆有空座位和可供行李盛放)的车辆。并将这些车辆以及若乘坐这班车辆应等待的时间和所需的金额等信息进行展示,以供乘客选择。
	\par\indent 乘客进入“选择和确认模块”。此时乘客可以根据系统上提示的拼车信息做出自己的选择,主要是对等待时间、金额、车辆行程路线以及所花费的时间等进行权衡。当乘客确定了某一具体的车辆后即可进行确认,确认完毕后,系统将用户此次选择的信息进行插入记录,并以此生成乘客等待倒计时以及等待号、将即将到站和所需等待的乘客基本信息展示在候车室内。
	\par\indent 乘客确认完毕后,系统会进入支付状态,此时乘客可以选择线上或线下进行支付。当乘客完成支付模块后,就可处于等待和被提醒乘车状态。
	\par\indent 当车辆到达之前,系统会通过微信短信和大厅显示器提醒乘客准备乘车。当车辆到达后,用户即可乘车以及进行本次拼车的信息反馈。反馈内容包括:用户体验、对司机以及驾驶状态的评价、紧急情况下的报警处理等。
\section{工作原理}
	本拼车系统是软硬件相结合,通过建立乘客信息以及车辆信息的数据库进行信息匹配,用高级语言进行实现。最终将以APP和小程序的形式供乘客使用。
